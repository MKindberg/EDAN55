\documentclass{tufte-handout}
\usepackage{amsmath,amsthm}

\usepackage{booktabs}
\usepackage{graphicx}
\usepackage[separate-uncertainty]{siunitx}
\usepackage{tikz}

\usepackage{amstext} % for \text macro
\usepackage{array}   % for \newcolumntype macro
\newcolumntype{L}{>{$}l<{$}} % math-mode version of "l" column type


\newtheorem{claim}{Claim}[section]
\title{\sf Marking Trees}
\date{}
\begin{document}
\section{Lab Report: Marking Trees}


by Marcus Kindberg

\subsection{Results}

For $i\in\{1,2,3\}$, the number of rounds $R_i$ spent until the tree
is completely marked in process $i$ is given in the following table.
The table shows the result of $1000$ repeated
trails.

\medskip\noindent
\begin{tabular}{
    S[table-format = 7]
    S[table-format = 1.1(1)e1]
    S[table-format = 1.4(1)e1]
    S[table-format = 1.0 e1]
  } 
% WARNING: This table the (brilliant) siunitx package.
% This allows typesetting of nicely aligned numbers.
% If this is too much to absorb, just use a normal Latex table.
% (Or do the table in another tool, export as PDF, and include it.)
% Or do the whole report in your favourite word processor instead.
\toprule

{ $N$ } & { $R_1$ } & {$R_2$} & {$R_3$} \\\midrule
3 & 2.5 \pm 0.9 & 2 & 2 \\
7 & 7 \pm 2 & 4.5 \pm 0.6 & 4  \\
15 & 1.7 \pm 7 e1 & 1.0 \pm 0.2 e1 & 8 \\
31 & 4 \pm 2 e1 & 2.3 \pm 0.3 e1 & 2 e1 \\
63 & 1.1 \pm 0.4 e2 & 5.1 \pm 0.6 e1 & 3 e1 \\
127 & 2.7 \pm 0.8 e2 & 1.08 \pm 0.09 e2 & 6 e1 \\
255 & 6 \pm 2 e2 & 2.3 \pm 0.1 e2 & 1   e2 \\
511 & 1.4 \pm 0.3 e3 & 4.7 \pm 0.2 e2 & 3   e2 \\
1023 & 3.2 \pm 0.7 e3 & 9.7 \pm 0.3 e2 & 5   e2 \\
$\vdots$ & $\vdots$ & $\vdots$ & $\vdots$ \\
524287 & 3.2 \pm 0.3 e6 & 5.230 \pm 0.006 e5 & 3   e5 \\
1048575 & 6.8\pm 0.7 e6 & 1.0468 \pm 0.0009 e6 & 5   e5 \\ \bottomrule
\end{tabular}

\subsection{Analysis}

Our experimental data indicates that $\mathbf E [R_1]$ is linearithmic,
while $\mathbf E[R_2]$ and $\mathbf E[R_3]$ are linear.

Theoretically, the behaviour of $R_1$ can be explained as follows: 

The problem is an "easier" version of the coupon collecting problem, which therefore can be used as an upper bound. 
An easier version of this problem would be if we, when we choose a node to mark, also mark its sibling and its  parent. This would be equal to a version of the coupon collecting problem where you get three cards each round instead of one. 
Since both the upper and lower bound can be reduced to coupon collecting, which is linearithmic, the problem itself is linearithmic too.

\end{document}